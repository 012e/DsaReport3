\chapter{Câu 2: Các bài tập nâng cao}

\section{Câu 2.1:}
\subsection{Ý tưởng}
\subsection{Cài đặt}
\subsection{Demo}

\section{Câu 2.2:}
\subsection{Ý tưởng}
\subsection{Cài đặt}
\subsection{Demo}

\section{Câu 2.3: Khóa số}

\paragraph{Để tăng độ an toàn chống hiện tượng cướp ngân hàng ngày càng phổ biến
người ta dùng khóa số với mã mở khóa đơn giản nhưng rất hiệu quả. Trên cửa
ra vào hiển thị một xâu khá dài các ký tự số. Các chữ số có thể di chuyển đổi
chổ cho nhau hoặc bị xóa. Muốn mở khóa người ta phải di chuyển các chữ số
và trong trường hợp cần thiết – xóa vài chữ số để nhận được xâu lớn nhất thỏa
mãn điều kiện đã cài đặt. Điều kiện này được thay đổi thường xuyên. Hôm nay
điều kiện đó là \textit{``Số nhận được phải chia hết cho 3''}. Số nhận được có thể bắt
đầu bằng các chữ số 0. Xâu ``000'' sẽ lớn hơn xâu ``00''.
Hãy xác định khóa mở cửa.}

\subsection{Ý tưởng}
Ta sẽ giải quyết bài toán trên modulo 3 - tức là đồng dư mod 3 gồm 0, 1 và 2
(Cụ thể hơn modulo 3 ở đây là phép chia lấy dư cho 3. VD $5 \equiv 2 \pmod 3$, ta nói 5 đồng dư 2 mod 3
hay 5 chia 3 dư 2).\\
Phân hoạch password ban đầu thành 10 phần tương ứng với 10 chữ số từ 0 đến 9.\\
Ta sẽ làm việc với số các chữ số xuất hiện trong password
    

\subsection{Thuật toán}
\begin{enumerate}
    \item Đánh dấu: đếm số các chữ số i có trong password (với i là các chữ số từ 0 đến 9)
        - Thuật toán xếp: trên ý tưởng đã phân hoạch password thành tập ``số các chữ số'', để sắp lại password
        lớn nhất ta xét các chữ số từ 9 về 0 và xếp tất cả chúng vào password theo thứ tự.
        VD: có 3 số 8, 2 số 1 và 1 số 0. Ta xếp theo thứ tự 888110
    \item Tính tổng Sum các chữ số để xét modulo mod 3 của số đó
        Sẽ có 3 trường hợp mod 3 của Sum
    \item Xét trường hợp:
    \begin{enumerate}
        \item Nếu $Sum \equiv 0 \pmod 3$, ta dùng thuật toán ``Xếp'' xếp lại password
        \item Nếu $Sum \equiv 1 \pmod 3$
        \begin{itemize}
            \item Để thu được password chia hết cho 3 ta sẽ bỏ đi 1 chữ số $\equiv 1 \pmod 3$ nhỏ nhất, nếu không
            có ta phải bỏ đi 2 chữ số $\equiv 2 \pmod $ nhỏ nhất
            \item Các chữ số $\equiv 0 \pmod 3$ sẽ không ảnh hưởng nên được giữ lại. Nếu không có số để bỏ hoặc bỏ hết
            tất cả chữ số tức là không có password thoả.\\
            \qquad $\Rightarrow$ Dùng thuật toán ``Xếp'' lại password
        \end{itemize}
        \item Tương tự nếu $Sum \equiv 2 \pmod 3$
        \begin{itemize}
            \item Ta sẽ bỏ đi 1 chữ số $\equiv 2 \pmod 3$ nhỏ nhất, nếu không có ta phải bỏ đi 2 chữ số $\equiv 1 \pmod 3$
            nhỏ nhất
            \item Nếu không có số để bỏ hoặc bỏ hết tất cả chữ số tức là không có password thoả.
            \\ \qquad $\Rightarrow$ dùng thuật toán ``Xếp'' lại password
        \end{itemize}
    \end{enumerate}
\end{enumerate}

\textbf{Độ phức tạp}: $O(n)$

\subsection{Cài đặt}
\begin{minted}[]{c++}
#include <bits/stdc++.h>

using namespace std;

void Xep(string &password, int a[], const int &n) {
    password.clear();
    for (int i = 9; i >= 0; i--)
        password += string(a[i], i + '0');
}

void DanhDau(const string &password, int a[]) {
    for (int i = 0; i < password.length(); i++) {
        ++a[int(password[i]) - 48];
    }
}

bool LoaiBo1(int a[], int &n) {
    if (a[1] + a[4] + a[7] >= 1) {
        if (a[1] > 0)
            --a[1];
        else if (a[4] > 0)
            --a[4];
        else
            --a[7];
        --n;
        if (n > 0)
            return true;
    } else if (a[2] + a[5] + a[8] >= 2) {
        for (int k = 0; k < 2; k++) {
            if (a[2] > 0)
                --a[2];
            else if (a[5] > 0)
                --a[5];
            else
                --a[8];
            --n;
        }
        if (n > 0)
            return true;
    }
    return false;
}

bool LoaiBo2(int a[], int &n) {
    if (a[2] + a[5] + a[8] >= 1) {
        if (a[2] > 0)
            --a[2];
        else if (a[5] > 0)
            --a[5];
        else
            --a[8];
        --n;
        if (n > 0)
            return true;
    } else if (a[1] + a[4] + a[7] >= 2) {
        for (int k = 0; k < 2; k++) {
            if (a[1] > 0)
                --a[1];
            else if (a[4] > 0)
                --a[4];
            else
                --a[7];
            --n;
        }
        if (n > 0)
            return true;
    }
    return false;
}

void Check(string &password, int a[], const int &sum, int &n) {
    if (sum % 3 == 0) {
        Xep(password, a, n);
        cout << "Password is " << password << endl;
    } else if (sum % 3 == 1) {
        if (LoaiBo1(a, n)) {
            Xep(password, a, n);
            cout << "Password is " << password << endl;
        } else
        cout << "Password khong ton tai!" << endl;
    } else {
        if (LoaiBo2(a, n)) {
            Xep(password, a, n);
            cout << "Password is " << password << endl;
        } else
        cout << "Password khong ton tai!" << endl;
    }
}

int main() {
    string password;
    int a[10] = {};
    int sum = 0;
    getline(cin, password);
    int n = password.length();
    DanhDau(password, a);
    for (int i = 0; i < 10; i++)
        sum += (a[i] * i);
    Check(password, a, sum, n);
}
\end{minted}

\subsection{Demo}
Sau khi compile và chạy chương trình với \mintinline{shell}{g++ main.cpp -o main && ./main} thì với các
input sau đây thì chương trình xuất các output tương ứng là:
